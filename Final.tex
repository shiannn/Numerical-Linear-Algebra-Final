\documentclass[12pt]{article}
\usepackage{CJKutf8}
\usepackage{geometry}
\usepackage{listings}
\usepackage{amsmath}
\usepackage{commath}
\usepackage{graphicx}
\usepackage{url}

\lstset{frame=tb,
    language=Matlab}

\newgeometry{vmargin={6mm,10mm}, hmargin={10mm,10mm}}   
\begin{CJK}{UTF8}{bsmi}
\author{b05502087 王竑睿}
\date{}
\title{數值線性代數 Final}

\begin{document}
\maketitle

    \section{A function $f:R^{m \times n} \rightarrow R$ is unitarily invariant if}
    $$f(A) = f(UAV)$$
    \subsection*{for any A $\in R^{m\times n}$ and for any orthogonal U $\in R^{m\times n}$ and V $\in R^{m\times m}$}
    \subsection*{(a) Use the fact that the Frobenius norm is unitarily invariant to show that}
    $$\norm{A}_F = \sum_{i=1}^{r}\sigma_i^2$$
    \subsection*{where $\sigma_1\geq \sigma_2\geq \sigma_3\geq \ldots  \geq \sigma_r > 0$ are singular values of A, $r \leq \min(m,n)$}
    \subsection*{[solution:]}
        \[
            \begin{aligned}
                &\norm{A}_F\\
            =   &\norm{U \textstyle \sum V}_F \quad \text{(SVD分解)}\\
            =   &\norm{\textstyle \sum}_F   \quad \text{(Frobenius norm 是 unitarily invariant)}\\
            =   &\sqrt{\sum_{i=1}^{r}\sigma_i^2} \quad \text{(Frobenius norm : square root of the sum of squares of its elements)}\\
            \end{aligned}    
        \]

    \subsection*{(b) Show that the truncated SVD $A_k = U_k \sum_k V_k^T$, $k\leq r$, provides the optimal
    rank-k approximation of A in the sense defined by}
    $$\norm{A-A_k}_F = \min_{B,rank(B)=k}\norm{A-B}_F$$
    \subsection*{where $\norm{A-A_k}_F = \sqrt{\sum_{i=k+1}^{r}\sigma_i^2}$}
    \subsection*{[solution:]}
    \begin{itemize}
        \item 令B是所有rank-k的matrices,令V是B的列向量所張出的空間,因此V的維度為k
        \item 若要minimize $\norm{A-B}^2_F$, B的各列向量應該要是A投影到V空間上的結果
            \subitem 否則只要把B的row換成A的對應列在V上的投影即可得到更小的結果
        \item 因為$A_k=U_k\textstyle \sum V_k^T$能夠最小化A的列向量到任何k維子空間的平方距離
            \subitem 因此$A_k$即為所求
    \end{itemize}
    \text{推導$\norm{A-A_k}_F = \sqrt{\sum_{i=k+1}^{r}\sigma_i^2}$}\\
    \text{令$u_k$為$U$的各行向量,$v_k$為$V$的各行向量}\\
        \[
            \begin{aligned}
                &A-A_k\\
            =   &\sum_{i=1}^{r}\sigma_iu_iv_i' - \sum_{i=1}^{k}\sigma_iu_iv_i'\\
            =   &\sum_{i=k+1}^{r}\sigma_iu_iv_i'\\
    \Rightarrow &\norm{A-A_k}_F\\
            =   &\norm{\sum_{i=k+1}^{r}\sigma_iu_iv_i'}_F\\
            =   &\norm{U'\sum_{i=k+1}^{r}\sigma_iu_iv_i' V}_F \quad \text{(Frobenius norm 是 unitarily invariant)}\\
            =   &\norm{\sum_{i=k+1}^{r}\sigma_ie_ie_i'}_F \quad \text{(Unitary Matrix 乘入變成單位向量)}\\
            =   &\sqrt{\sum_{i=k+1}^{r}\sigma_i^2}\\
            \end{aligned}        
        \]

    \subsection*{(c) Following part (b), what is $\norm{A - A_k}_2$ and what is $\norm{A - A_k}_*$ ?}
    \subsection*{[solution: $\norm{A-A_k}_2$]}
        \[
            \begin{aligned}
                &\norm{(A-A_k)v}_2  \quad \text{(v是$A-A_k$的top singular vector)}\\
            =   &\norm{\sum_{i=k+1}^{r}\sigma_iu_iv_i'\sum_{j=k+1}^{r}\alpha_{j}v_j}_2 \quad \text{將$v_j$利用$(v_1,v_2,...,v_r)$分解}\\
            =   &\norm{\sum_{i=k+1}^{r}\alpha_i\sigma_iu_iv_i'v_i}\\
            =   &\norm{\sum_{i=k+1}^{r}\alpha_i\sigma_iu_i} \quad \text{(v是單位向量)}\\
            =   &\sqrt{\sum_{i=k+1}^{r}\alpha_i^2\sigma_i^2} \quad \text{(向量的2-norm)}\\
    \Rightarrow &\text{依據norm的定義,要找一個v使這個2-norm最大}\\
                &\text{因為$\abs{v}^2 = \sum_{i=1}^{r}\alpha_{i}^2 = 1$}\\
                &\text{又因為$\sigma_{k+1}$是$\sigma_{k+1}$到$\sigma_{r}$中最大的。}\\
                &\text{所以,讓$\alpha_{k+1}=1$,其餘為0,可得到最大值}\\
            \end{aligned}
        \]
        {\LARGE $$\sigma_{k+1}$$}
    
    \subsection*{[solution: $\norm{A-A_k}_*$]}
        \[
            \begin{aligned}
                &\norm{A-A_k}_*  \\
            =   &\norm{\sum_{i=k+1}^{r}\sigma_iu_iv_i'}_* \\
            =   &\sum_{i=k+1}^{r}\sigma_i\\
                &\text{Claim that:}\\
                &\norm{A-A_k}_* = \sum_{i=k+1}^{r}\sigma_i \leq \norm{A-B_k}_* \quad \text{(if $B_k=XY'$ has k columns)}\\
                &\text{By triangle inequality, if $A=A'+A''$:}\\
                &\text{then $\sigma_1(A)\leq\sigma_1(A')+\sigma_1(A'')$}\\
    \Rightarrow &\sigma_i(A')+\sigma_j(A'')\\
            =   &\sigma_1(A'-A'_{i-1})+\sigma_1(A'-A'_{j-1})\\
        \geq    &\sigma_1(A'-A'_{i-1}-A''_{j-1})\\
        \geq    &\sigma_1(A-A_{i+j-2})    \quad \text{(since $rank(A'_{i-1}+A''_{j-1})\leq rank(A_{i+j-2})$)}\\
            =   &\sigma_{i+j-1}(A)\\
                &\text{[since $\sigma_{k+1}(B_k)=0$, when $A'=A-B_k$ and $A''=B_k$, we conclude that for $i\geq 1$, $j=k+1$]}\\
                &\sigma_i(A-B_k)\geq\sigma_{k+i}(A)\\
                &\text{Therefore,}\\
                &\norm{A-B_k}_* = \sum_{i=1}^{n}\sigma_{i}(A-B_k)\geq \sum_{i=k+1}^{n}\sigma_{i}(A) = \norm{A-A_k}_*
            \end{aligned}
        \]

    \section{Given the values of the Laplace transform at points $s_j, 0 < s_1 < ...< s_n < \infty$,
            we want to estimate the function f.}
    \begin{itemize}
        \item \textbf{本題目中使用到的Gauss Quadrature參考: }
            \subitem \url{$https://github.com/sfstoolbox/sfs-matlab/blob/master/SFS_general/legpts.m$}
        \item 利用以下code,取得矩陣A, y以及true signal下的xtrue
    \end{itemize}
    \begin{lstlisting}
        function Xtrue = getTrueX(T)
            N = size(T);
            Xtrue = zeros(N);
            for i = 1:N 
                t = T(i);
                if(t<=1)
                    Xtrue(i) = t;
                elseif(1<=t && t<3)
                    Xtrue(i) = 3/2-t/2;
                elseif(3<=t)
                    Xtrue(i) = 0;
                end
            end
        end
        function A = getA(W,S,T)
            J = size(S,1);
            K = size(T,1);
            A = zeros(J,K);
            for j = 1:J
                for k = 1:K
                    A(j,k) = W(k)*exp((-1)*S(j)*T(k));
                end
            end
        end
        function Y = getY(S)
            N = size(S);
            Y = zeros(N);
            for i = 1:N 
                Y(i) = getLf(S(i));
            end
        end
        function S = log_dis(N)
            S = zeros(N,1);
            for j = 1:N 
                temp = (-1 + (j-1)/20)*log(10);
                S(j) = exp(temp);
            end
        end
        function Lf = getLf(s)
            Lf = (2-3*exp((-1)*s)+exp((-3)*s))/(2*s*s);
        end
    \end{lstlisting}
    \subsection*{(a) To appreciate the ill-posedness of this problem, try to estimate the values
        $x_j = f(t_j)$ by direct solution of the system Ax = y, using the backslash
        command in Matlab, using analytically known data with no artificial error
        added to it}
    \subsection*{[Solution:]}
    利用以下code,進行matlab的左除運算,並且作圖
    \begin{lstlisting}
        function solve()
            N = 40;
            S = log_dis(N); %do logarithmically distributed to get sj
            Y = getY(S);
            [T W] = legpts(N,[0,5],'GW'); %do quad to get wk, tk
            A = getA(W,S,T);%use s t to get A
            Xcal = A\Y;%Ax=y
            Xtrue = getTrueX(T); %true f(t)
            plot([1:N],Xcal);
            hold on
            plot([1:N],Xtrue);
            xlabel("elements i'th");
            ylabel("value of Xcal(i) and Xtrue(i)");
            legend('Xcal','Xtrue')
            norm(Xcal-Xtrue,2)
        end
    \end{lstlisting}
    norm(Xcal-Xtrue,2)的結果為2.441640785552484e+03 $\approx$ 2441.64\\
    \includegraphics[scale=0.75]{backslash.jpg}

    \subsection*{(b) Calculate the SVD of A and use TSVD to approximate the solution of this
        problem. How many singular values you need to get an error of the solution to
        $O(10^{-d})$, for some $d > 0$}
    \subsection*{[Solution:]}
    \begin{itemize}
        \item 利用以下code,進行Truncated SVD並作圖
        \item matlab svd會對singular value進行排序,大的靠左上,因此我們從右下開始truncate
    \end{itemize}
    \begin{lstlisting}
        function TSVD()
            %do quad to get wk, tk
            %do logarithmically distributed to get sj
            N = 40;
            S = log_dis(N);
            Y = getY(S);
            [T W] = legpts(N,[0,5],'FAST');
            %use s t to get A
            A = getA(W,S,T);
            [U,S,V] = svd(A);
            PLOTX = [];
            PLOTY = [];
            for singuNum = 1:N
                Snew = S;
                for i=1:N 
                    if(i>singuNum)
                        Snew(i,i) = 0;
                    else 
                        Snew(i,i) = 1./Snew(i,i);
                    end
                end
                Strun = zeros(N,N);
                for i=1:N 
                    if(Snew(i,i)~=0)
                        Strun(i,i) = Snew(i,i);
                    end
                end
                Xcal = V*Strun*U'*Y;
                Xtrue = getTrueX(T);
                PLOTX = [PLOTX, size(nonzeros(diag(Snew)),1)];
                PLOTY = [PLOTY, norm(Xcal - Xtrue,2)];
            end
            plot(PLOTX,PLOTY);
            xlabel('Number of singular values')
            ylabel('2-norm error between Xtrue and Xcal')
        end
    \end{lstlisting}
    \includegraphics[scale=0.8]{trun.jpg}
    \begin{itemize}
        \item 由圖可知,在使用21個singular value (truncate掉19個)時,就能夠讓error降到$10^{-1}$以下
        \item 2-norm error約為0.226246062159252 $\approx$ 0.2262
    \end{itemize}
    
    \newpage
    \subsection*{(c) Add a tiny random error to the data, and estimate x from the noisy data using
        the Tikhonov regularization}
            $$x_\delta = argmin_x(\norm{Ax-y}^2_2+\delta^2\norm{x}^2_2)$$
    \subsection*{Try different values of the regularization parameter $\delta$, and show the estimates
        of the solutions. In all the cases considered here, be sure to make suitable plots
        of solutions or erros, and give comments of your findings.}
    \begin{itemize}
        \item 利用以下code,對y加入random noise。其中noise是落在[-1e-4,1e-4]間的均勻分佈
        \item 進行不同$\delta$值的Tikhonov regularization。並進行作圖
    \end{itemize}
    \begin{lstlisting}
        function Tik()
            N = 40;
            S = log_dis(N);
            Y = getY(S);
            %add random noise
            rad = 1e-4
            noise = (-1)*rad + 2*rad*rand(N,1)
            Ynoise = Y+noise;
        
            [T W] = legpts(N,[0,5],'GW');
            %use s t to get A
            A = getA(W,S,T);
            %use normal equation to get Xcal
            for pw = 2:5
                figure(pw);
                delta = 10^(-pw);
                I = eye(N);
                Aplus = (A'*A+delta*delta*I);
                B = A'*Ynoise;
                %Aplus*X = B
                Xcal = Aplus \ B;
                Xtrue = getTrueX(T);
        
                plot([1:N],Xcal);
                hold on
                plot([1:N],Xtrue);
                %title('\delta')
                title(['\delta = ',num2str(delta)])
                xlabel("elements i'th");
                ylabel("value of Xcal(i) and Xtrue(i)");
                legend('Xcal','Xtrue');
                norm(Xcal-Xtrue,2)
            end
        end
    \end{lstlisting}
    \includegraphics[scale=0.52]{tk-2.jpg}
    \includegraphics[scale=0.52]{tk-3.jpg}\\
    \includegraphics[scale=0.52]{tk-4.jpg}
    \includegraphics[scale=0.52]{tk-5.jpg}
    \begin{itemize}
        \item 上四圖為40維solution Xcal 與 true signal Xtrue 各個維度的值
        \item $\delta$為1e-2, 1e-3, 1e-4, 1e-5的2-norm error 分別為
            \subitem 0.246113880621484 $\approx$ 0.2461
            \subitem 0.138483848796576 $\approx$ 0.1385
            \subitem 0.435398463354816 $\approx$ 0.4354
            \subitem 2.785405979120068 $\approx$ 2.7854
        \item $\delta$為1e-3時,有最低的2-norm error
    \end{itemize}
\end{CJK}
\end{document}